\documentclass[a4paper,12pt]{ctexart} % 使用ctexart类,支持中文
\usepackage{geometry} % 页边距设置
\geometry{left=2.5cm, right=2.5cm, top=2.5cm, bottom=2.5cm}

\usepackage{fancyhdr} % 页眉页脚设置
\usepackage{setspace} % 设置行距
\usepackage{indentfirst} % 首行缩进
\usepackage{array} % 用于表格格式设置
\usepackage{titlesec} % 自定义标题格式
\usepackage{graphicx} % 插入图片
\usepackage{float}
\usepackage{listings}
\usepackage{xcolor}


\pagestyle{fancy} % 页脚页码居中
\fancyhf{} % 清除所有默认设置
\renewcommand{\headrulewidth}{0pt} % 删除页眉线
\fancyfoot[C]{\thepage} % 页脚页码居中


% 自定义加粗文字格式(四级标题)
\newcommand{\subsubsubsection}[1]{%
    \par\vspace{0.5\baselineskip} % 段前0.5行距
    \noindent\hspace{1.25em}{\zihao{4}\heiti #1} 
    \par\vspace{0.5\baselineskip} % 段后0.5行距
}

% 设置脚注符号为① ② ③ ...
\renewcommand{\thefootnote}{\textcircled{\arabic{footnote}}}

% 代码块设置
\lstset{ 
    language = C,
    numbers=left,
    keywordstyle=\color{blue!70},
    commentstyle=\color{red!50!green!50!blue!50},
    frame=shadowbox,
    rulesepcolor=\color{red!20!green!20!blue!20},
    basicstyle=\ttfamily,
    showstringspaces=false
}



% 设置封面标题格式
\newcommand{\makecover}{
    \begin{center}
        % 设置标题部分
        ~\\
        ~\\
        {\heiti\zihao{-0} 山东大学}\\[0.5cm]
        {\heiti\zihao{-0} 网络空间安全学院}\\[3cm]
        {\heiti\zihao{-0} 《XXXXXXX》}\\[1cm]
        {\heiti\zihao{-0} 实验报告}\\[4.5cm]

        % 设置表格部分
        \begin{tabular}{>{\heiti\zihao{-3}}r p{10cm}}
            学生姓名 & \underline{\makebox[10cm][c]{XXX(学号:YYYY)}} \\[0.35cm]
             & \underline{\makebox[10cm][c]{XXX(学号:YYYY)}} \\[0.35cm]
             & \underline{\makebox[10cm][c]{XXX(学号:YYYY)}} \\[0.35cm]
            指导教师 & \underline{\makebox[10cm][c]{XXXXXXXXXXXXXXXXXXXXXX}} \\[0.35cm]
            学\hspace{2em}院 & \underline{\makebox[10cm][c]{XXXXXXXXXXXXXXXXXXXXX}} \\[0.35cm]
            专业班级 & \underline{\makebox[10cm][c]{XXXXXXXXXXXXXXXXXX}} \\[0.35cm]
            完成时间 & \underline{\makebox[10cm][c]{XXXX年XX月XX日}} \\
        \end{tabular}
    \end{center}
    \thispagestyle{empty} % 封面页不显示页码
    \newpage % 分页
}

% 设置标题格式
\ctexset{
    section = {
        format+ = \zihao{-2}\heiti,
        name = {}, % 标题前缀为空
        % number = {}, % 取消编号
        beforeskip = 0.5\baselineskip,
        afterskip = 0.5\baselineskip,
        indent = 0em, % 标题左对齐,且无缩进
    },
    subsection = {
        format+ = \zihao{3}\heiti,
        name = {}, % 标题前缀为空
        % number = {}, % 取消编号
        beforeskip = 0.5\baselineskip,
        afterskip = 0.5\baselineskip,
        indent = 0em, % 一级标题左对齐,且无缩进
    },
    subsubsection = {
        format+ = \zihao{-3}\heiti,
        name = {}, % 标题前缀为空
        % number = {}, % 取消编号
        beforeskip = 0.5\baselineskip,
        afterskip = 0.5\baselineskip,
        indent = 0em, % 二级标题左对齐,且无缩进
    },
}

% 设置正文格式
\setlength{\parindent}{2em} % 首行缩进2字符
\setlength{\parskip}{0pt} % 段前段后0行

% 设置正文字体为宋体,小四,左右缩进0字符
\renewcommand{\normalsize}{\fontsize{12pt}{20pt}\selectfont} % 小四,行距20磅
\setlength{\leftskip}{0pt} % 左缩进0字符
\setlength{\rightskip}{0pt} % 右缩进0字符

% 正文部分
\begin{document}

% 封面
\makecover


\newpage
\tableofcontents
\setcounter{page}{0}
% \thispagestyle{empty}
\pagenumbering{Roman} %设置罗马数字页码
\clearpage
\pagenumbering{arabic} %设置阿拉伯数字页码


% 实验报告正文部分
\section{标题}
正文内容。请在此处撰写您的实验报告内容。请在此处撰写您的实验报告内容。

最后最好在本地编译一遍,overleaf的黑体比windows自带的黑体要粗一些

\subsection{实验目标}

\subsection{实验原理}

\subsection{实验设置}

\subsection{实验步骤及实验结果}

\subsection{一级标题}
更多内容。行内代码 \verb|sudo rm -rf /*|

\subsubsection{二级标题}
进一步的内容。信息技术广泛应用和网络空间兴起发展,极大促进了经济社会繁荣进步,同时也带来了新的安全风险和挑战。网络空间安全(以下称网络安全)事关人类共同利益,事关世界和平与发展,事关各国国家安全。维护我国网络安全是协调推进全面建成小康社会、全面深化改革、全面依法治国、全面从严治党战略布局的重要举措,是实现“两个一百年”奋斗目标、实现中华民族伟大复兴中国梦的重要保障。"这是一个直接引用的例子。"①\footnote{XX,XX,XX著,《XXX》[M],北京:中国人民大学出版社,2010年4月,第10页。}


\subsubsubsection{三级标题}
更多内容

这里是一张图片
\begin{figure}[H]
    \centering
    \includegraphics[width=0.5\linewidth]{images/HASHTEAM.jpg}
    \caption{hashteam}
    \label{fig:enter-label}
\end{figure}



\clearpage
% 参考文献部分
\begin{thebibliography}{99} % 参考文献环境,99代表标签宽度
\addcontentsline{toc}{section}{参考文献}
\zihao{-4} % 调整参考文献字体大小为“小四”


\bibitem{ref1} XX,XX,XX著,《XXX》[M],北京:中国人民大学出版社,2010年4月。
\bibitem{ref2} XX,XX著,XX,XX 译,《XXX》[M],上海:三联出版社,2010年4月。
\bibitem{ref3} XX,《XXX》[D],天津:南开大学硕士(博士)学位论文,2008年6月。
\bibitem{ref4} XX,XX,XX,《XXX》[J],经济研究,2008年6月,第XX-XX页。

\end{thebibliography}
\clearpage
\section*{附录}
\addcontentsline{toc}{section}{附录}


\lstset{language=sh}
\begin{lstlisting}
ping 127.0.0.1
\end{lstlisting}

\lstset{language=Python}
\begin{lstlisting}
#!/usr/local/bin/python3
inp = input("> ")

for i, v in enumerate(inp):
    if not (ord(v) < 128 and i % 2 == ord(v) % 2):
        print('bad')
        exit()

eval(inp)
\end{lstlisting}

\end{document}
